%\input logo

%\vspace*{-3cm}

%\begin{figure}[h]
%\leavevmode
%\begin{minipage}[t]{\textwidth}
%\includegraphics[1cm,1cm][3cm,3cm]{logo-ufrpe.bmp}
%\end{minipage}
%\end{figure}



\vspace*{-2cm}
{\bf
\begin{center}
{\large
\hspace*{0cm}Universidade de S�o Paulo} \\
\hspace*{0cm}Escola Polit�cnica \\
\hspace*{0cm}Curso de Engenharia de Computa��o Cooperativo  \\
\end{center}}
% se voc� souber meter o logo da POLI aqui, ficaria legal...
\vspace{4.0cm}
\noindent
\begin{center}
{\Large \bf Discretiza��o do modelo do P�ndulo} \\[3cm]
{\Large Bruno Andreetto, bruno.andreetto@usp.br}\\[6mm]
{\Large Daniel Trigo Rizzuto, daniel.rizutto@usp.br}\\[6mm]
{\Large .%MAP3122 - M�todos Num�ricos
}\\[3.0cm]
\end{center}

{\raggedleft
\begin{minipage}[t]{8.0cm}
\setlength{\baselineskip}{0.25in}
RELAT�RIO apresentado ao Professor Alexandre Roma do MAP/IME-USP 
como atividade da disciplina MAP3122 - M�todos Num�ricos.
\end{minipage}\\[2cm]}
% o logo tamb�m ficaria legal aqui legal...
\vspace{2cm}
{\center S�o Paulo - SP \\[3mm]
31/03/2016 \\}


\newpage